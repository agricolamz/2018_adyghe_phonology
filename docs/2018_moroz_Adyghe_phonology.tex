\documentclass[russian,]{book}
\usepackage{lmodern}
\usepackage{amssymb,amsmath}
\usepackage{ifxetex,ifluatex}
\usepackage{fixltx2e} % provides \textsubscript
\ifnum 0\ifxetex 1\fi\ifluatex 1\fi=0 % if pdftex
  \usepackage[T1]{fontenc}
  \usepackage[utf8]{inputenc}
\else % if luatex or xelatex
  \ifxetex
    \usepackage{mathspec}
  \else
    \usepackage{fontspec}
  \fi
  \defaultfontfeatures{Ligatures=TeX,Scale=MatchLowercase}
\fi
% use upquote if available, for straight quotes in verbatim environments
\IfFileExists{upquote.sty}{\usepackage{upquote}}{}
% use microtype if available
\IfFileExists{microtype.sty}{%
\usepackage{microtype}
\UseMicrotypeSet[protrusion]{basicmath} % disable protrusion for tt fonts
}{}
\usepackage[margin=1in]{geometry}
\usepackage{hyperref}
\hypersetup{unicode=true,
            pdftitle={Фонология адыгских языков},
            pdfauthor={Г. А. Мороз},
            pdfborder={0 0 0},
            breaklinks=true}
\urlstyle{same}  % don't use monospace font for urls
\ifnum 0\ifxetex 1\fi\ifluatex 1\fi=0 % if pdftex
  \usepackage[shorthands=off,main=russian]{babel}
\else
  \usepackage{polyglossia}
  \setmainlanguage[]{russian}
\fi
\usepackage{natbib}
\bibliographystyle{apalike}
\usepackage{longtable,booktabs}
\usepackage{graphicx,grffile}
\makeatletter
\def\maxwidth{\ifdim\Gin@nat@width>\linewidth\linewidth\else\Gin@nat@width\fi}
\def\maxheight{\ifdim\Gin@nat@height>\textheight\textheight\else\Gin@nat@height\fi}
\makeatother
% Scale images if necessary, so that they will not overflow the page
% margins by default, and it is still possible to overwrite the defaults
% using explicit options in \includegraphics[width, height, ...]{}
\setkeys{Gin}{width=\maxwidth,height=\maxheight,keepaspectratio}
\IfFileExists{parskip.sty}{%
\usepackage{parskip}
}{% else
\setlength{\parindent}{0pt}
\setlength{\parskip}{6pt plus 2pt minus 1pt}
}
\setlength{\emergencystretch}{3em}  % prevent overfull lines
\providecommand{\tightlist}{%
  \setlength{\itemsep}{0pt}\setlength{\parskip}{0pt}}
\setcounter{secnumdepth}{5}
% Redefines (sub)paragraphs to behave more like sections
\ifx\paragraph\undefined\else
\let\oldparagraph\paragraph
\renewcommand{\paragraph}[1]{\oldparagraph{#1}\mbox{}}
\fi
\ifx\subparagraph\undefined\else
\let\oldsubparagraph\subparagraph
\renewcommand{\subparagraph}[1]{\oldsubparagraph{#1}\mbox{}}
\fi

%%% Use protect on footnotes to avoid problems with footnotes in titles
\let\rmarkdownfootnote\footnote%
\def\footnote{\protect\rmarkdownfootnote}

%%% Change title format to be more compact
\usepackage{titling}

% Create subtitle command for use in maketitle
\newcommand{\subtitle}[1]{
  \posttitle{
    \begin{center}\large#1\end{center}
    }
}

\setlength{\droptitle}{-2em}

  \title{Фонология адыгских языков}
    \pretitle{\vspace{\droptitle}\centering\huge}
  \posttitle{\par}
    \author{Г. А. Мороз}
    \preauthor{\centering\large\emph}
  \postauthor{\par}
      \predate{\centering\large\emph}
  \postdate{\par}
    \date{2018}

\usepackage{fontspec}
\defaultfontfeatures{Ligatures={TeX},Renderer=Basic}
\setmainfont[Ligatures={TeX,Historic},
SmallCapsFont={Brill},
SmallCapsFeatures={Letters=SmallCaps}]{Brill} 
\newfontfamily{\cyrillicfonttt}{Brill}
\usepackage{indentfirst}
\setlength{\parindent}{15pt}
\setlength{\parskip}{0em}
\usepackage{booktabs}
\usepackage{amsthm}
\makeatletter
\def\thm@space@setup{%
  \thm@preskip=8pt plus 2pt minus 4pt
  \thm@postskip=\thm@preskip
}
\makeatother

\begin{document}
\maketitle

{
\setcounter{tocdepth}{1}
\tableofcontents
}
\chapter{Введение}\label{intro}

Данное исследование посвящено различным аспектам фонологии и фонетики
адыгских языков. Несмотря на то, что работ, посвященных тем или иным
аспектам данной темы, достаточно много, еще не было предпринято попыток
единообразного описания адыгской фонологии. Часто исследователи либо
рассматривают частные проблемы тех или иных адыгских идиомов, либо же
ограничиваются описанием литературных норм. В нашем исследовании мы
ставим себе задачу единообразного описания разных фонологических систем
всех адыгских идиомов, распространенных на территории РФ.

\textbf{Предметом исследования} настоящей диссертации послужили
сегментные и супрасегментные единицы адыгских идиомов.

\textbf{Объектом исследования} является лексика и фонетика адыгских
идиомов на территории РФ. Материалом исследования послужили лексический
материал и аудиозаписи, собранные в ходе полевых исследований в 21
адыгском ауле, а также материалы, представленные в опубликованных ранее
работах.

\textbf{Целью исследования} является единообразное описание фонологии
адыгских языков и уточнение имеющихся данных по отдельным адыгским
идиомам. Кроме того на основе собранного материала мы выясняем
лингвистическое расстояние между идиомами и соотношение результатов
полученные в ходе исследования с традиционным разделением адыгских
идиомов.

\textbf{Актуальность исследования} определяется отсутствием полного
единообразного описания адыгских идиомов, а работы, претендующие на
такую полноту, не учитывают особенностей ряда адыгских идиомов. Кроме
того, многие описания отдельных идиомов в значительной мере устарели. В
данном исследовании дается описание современного состояния многих
идиомов, которые часто отличаются от описаний, представленных в работах
по данной теме.

\textbf{Задачи исследования}:

\begin{itemize}
\tightlist
\item
  единообразное описание аспектов адыгской фонологии * инвентаря гласных
  * инвентаря согласных * системы ударения * структуры слога
\item
  кроме того в работе затрагиваются такие просодические особенности как
  скорость речи
\end{itemize}

\textbf{Научная новизна} диссертации состоит в создании единообразного
описания звуковых систем адыгских идиомов, во введении в научный оборот
новых данных фонетического и фонологического исследований адыгских
идиомов. Исследование лингвистического разнообразия идиомов позволяет
сделать шаг к анализу и классификации диалектного разнообразия адыгских
идиомов. Кроме того в работе приводятся результаты ряда экспериментов,
которые к адыгским языкам ранее не применялись.

\textbf{В теоретическом плане} диссертационное исследование вносит вклад
в понимание фонологического устройства адыгских языков. Данные
результаты могут послужить материалом для работ по фонетической и
фонологической типологии. Кроме того результаты работы дают материал для
исследований в области сравнительно-исторического языкознания. В ходе
исследования были обнаружены звуки, которые не были описаны ни в рамках
адыговедения, ни в рамках теоретической фонетики. Опыт анализа
лингвистического разнообразия позволяет оценить применимость алгоритмов
кластеризации в лингвистическом исследовании.

\textbf{Практическая значимость} работы заключается в возможности
использовать результаты работы при изучении адыгских языков в школах и
университетах, особенно носителям идиомов, не имеющих статуса
литературного языка. Сформулированные различия между идиомами, позволяют
уточнить школьные и университетские программы преподавания адыгских
языков.

\textbf{Основным материалом исследования} послужили лексический материал
и аудиозаписи фонетических анкет, собранные автором в 2011-2016 гг., а
также словари и другие опубликованные материалы адыгских языков.

\textbf{Апробация результатов исследования} прошла в форме докладов на
Научной студенческой конференции ``Полевые исследования студентов РГГУ''
(2011 г.), XII научных чтениях памяти Г. А. Ткаченко (2012 г.), Научной
студенческой конференции ``Полевые исследования студентов РГГУ'' (2012
г.), Первой конференции-школе ``Проблемы языка: взгляд молодых ученых''
(2012 г.), конференции ``Народная лингвистика: взгляд носителей языка на
язык'' (2012 г.), на Четвёртой конференции-школе ``Проблемы языка:
взгляд молодых ученых'' (2015 г.), а также в ходе ежегодной подготовки
студентов к летней экспедиционной практике, организованной Институтом
лингвистики Российского государственного гуманитарного университета и
Школой лингвистики Национального исследовательского университета
``Высшая школа экономики''.

\textbf{Диссертации} включает три главы, список литературы и шесть
приложений. Во Введении обосновывается актуальность диссертации,
формулируются цели исследования, описывается использованный материал,
характеризуется практическая и теоретическая значимость данной работы,
перечисляются положения, выносимые на защиту. В главе \ref{socio}
обосновывается понятие идиома, излагаются разные классификации адыгских
идиомов, приводившиеся ранее, а также описываются современное расселение
адыгов и социолингвистическая позиция адыгов в РФ и Турции. В главе
\ref{segments} рассматриваются орфографические системы адыгских идиомов,
предлагается практическая кириллическая транскрипция, описываются
адыгские вокалические системы, системы взрывных, фрикативных, аффрикат и
сонорных. В главе \ref{suprasegmentals} рассматриваются супрасегментные
единицы адыгских языков: ударение, структура слога и скорость речи. В
заключительной главе диссертации \ref{conclusions} излагаются основные
выводы. Заканчивается диссертация библиографическим списком и
приложениями.

\textbf{Положения, выносимые на защиту:}

\begin{itemize}
\tightlist
\item
  Вокализм адыгских языков на поверхностном уровне представляет собой
  треугольную систему, распространенную во многих языках мира, с той
  лишь особенностью, что в адыгских языках отсутствуют гласные верхнего
  подъема (i и u), а есть лишь гласные средне-верхнего подъема, которые
  точнее передаются следующими знаками Международного фонетического
  алфавита: ɪ и ʊ.
\item
  Образованные от основных гласных (a, ɐ, ə), вторичные гласные (i, e,
  o, u) ведут себя по-разному в адыгских языках: в адыгейских идиомах
  вторичные гласные фонологически можно интерпретировать как сочетание
  глайда j и w и ядерных гласных a, ɐ, ə, а в кабардино--черкес-ских
  идиомах вторичные гласные можнообразомобразом
  интерпретиобразомобразомровать как сочетания ядерных гласных и глайдов
  или же сочетание глайд -- гласный -- глайд.
\item
  В области консонантизма современные адыгские языки значительно
  отличаются от описаний середины XX века, что можно объяснить либо
  изменением языка, либо некорректностью некоторых описаний, либо же
  комбинацией этих двух утверждений.
\item
  В области ударения адыгские идиомы устроены по-разному. Ударение
  кабардино-черкесских языков можно описать при помощи двух правил,
  оперирующим морфонологическим понятием основы (ударение падает на
  последний слог основы, кроме случаев, когда последний слог основы ---
  открытый слог с гласным ɐ), и нескольких исключений. Ударение
  адыгейского языка носит достаточно вариативный характер и, возможно,
  не является обязательным для всех лексических единиц.
\item
  Слоговую структуру адыгских языков можно описать как
  (C)(C)(C)V(C)(C)(C), с одним лишь ограничением, что в консонантном
  кластере сонорные занимают ближайшую к ядру слога позицию.
\item
  Наиболее частотные консонантные кластеры могут встречаться как в
  инициали, так и в финали слога, однако для каждого конкретного случая
  существуют свои предпочтения, например, комбинации
  взрывной--фрикативный, фрикативный--фрикативный, фрикативный--взрывной
  тяготеют к позиции в инициали, а сочетание сонорный--взрывной
  встречается исключительно в финали.
\item
  Средняя скорость речи, измеренная на материале кубанского диаклекта
  кабардино--черкесского языка, составила 4.33 слогов в секунду с 95\%
  доверительным интервалом валом ±0.18, что меньше чем значения,
  полученные в исследованиях других языков.
\item
  Значение корреляции между скоростью речи и длительностью произносимой
  фразы, обнаруженной в некоторых работах по скорости речи, на нашем
  материале оказалось слишком маленькой (0.1252841).
\end{itemize}

\textbf{Публикации по теме диссертации:}
\citep{moroz12a, moroz12b, moroz12с, moroz14b, bagirokova15, lander15, moroz17},
в том числе рекомендованных ВАК: \citep{moroz14a, moroz15}.

\chapter{Обзор социолингвистической ситуации и истории классификации
адыгских идиомов}\label{socio}

\chapter{Сегментные единицы адыгских идиомов}\label{segments}

\chapter{Супрасегментные единицы в адыгских языках и скорость
речи}\label{suprasegmentals}

\chapter{Заключение}\label{conclusions}

\bibliography{bibliography.bib}


\end{document}
